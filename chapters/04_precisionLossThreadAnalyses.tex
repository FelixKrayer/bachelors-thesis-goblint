% !TeX root = ../main.tex
% Add the above to each chapter to make compiling the PDF easier in some editors.

\chapter{Combatting the Precision Loss of Thread Analyses}\label{chapter:precisionLossThreadAnalyses}
% local traces
  To track relational information between variables in multithreaded programs, M. Schwarz et al. propose a set of analyses specialized for this purpose in their paper "Clustered Relational Thread-Modular Abstract Interpretation with Local Traces"~\parencite{schwarz2023clustered}. In this chapter we take a closer look at one of these analyses and identify a potential loss of precision when this analysis is performed partially context sensitively. Furthermore, we propose an approach to reduce this precision loss in certain cases.
  \section{Theory}\label{sec:threadTheory}
  In Chapter 6 of their paper \parencite{schwarz2023clustered} they propose an analysis that identifies threads by their creation history. Among other things, this analysis helps to identify which threads are unique and which actions may or may not happen in parallel.\\
  As mentioned, threads are identified by their creation history, which is used as an ID to identify different threads. This history is a sequence of create edges starting with $\textsf{main}$. To prevent such a history to grow to infinity, they define the notion of non-unique thread IDs which may identify multiple threads each.\\
  Formally, the set of possible abstract thread IDs $\mathcal{I}^{\#}$ is $(\textsf{main}\cdot\mathcal{P}^*) \times 2^{\mathcal{P}}$, where $\mathcal{P}$ is the set of create edges and $\mathcal{P}^*$ a sequence of such edges. $\langle u, f \rangle \in \mathcal{P}$ refers to an outgoing edge from program point $u$ which creates a thread starting at $f$.
  In this notion, IDs of the form $(i, \emptyset) \in \mathcal{I}^{\#}$ are unique, while $(i, s) \in \mathcal{I}^{\#}$ are not unique if $s \neq \emptyset$.\\
  \\
  As mentioned, these abstract thread IDs are found by a dedicated thread ID analysis. In order to work correctly, this analysis needs to track the current thread ID as well as a set of create edges already encountered. Thus, the abstract domain for this analysis is $\mathbb{D}_\textsf{tID} = (\mathcal{I}^{\#} \times 2^{\mathcal{P}})$. An element $(id, es)$ of this domain is the product of the current thread ID $id$ and the set of encountered create edges $es$.\\
  The main part of the analysis now works as follows: Assume a thread-create edge $(u, create(f), v)$ is encountered with a state $(id, es)$. When the $id$ is already non-unique, the new thread is identified with a (possibly new) non-unique ID. In the other case $id$ is unique, i.e., $id = (i, \emptyset)$. Then the analysis checks whether the currently encountered edge was already encountered before, i.e., if it is present in the set of encountered edges $es$. If not, the new thread is identified with a new unique ID that is created by appending the current create edge to the first part of $id = (i, \emptyset)$. Otherwise, the new thread is identified with a non-unique ID.\\
  % TODO: Formally?
  It is worth mentioning, that the $\textsf{combine}^{\#}_\textsf{tID}$ function is implemented to ignore the caller state and propagate the callee return state for this analysis.
  \\
  As of now the thread ID analysis is always run context sensitively, i.e., $\mathbb{C}_\textsf{tID} = \mathbb{D}_\textsf{tID}$ and $\textsf{context}^{\#}_\textsf{tID}\ D = \textsf{enter}^{\#}_\textsf{tID}\ D$. Performing it context insensitively is not practical. The reason for that is leads to cases where a least upper bound of two states $(id_1, es_1)$ and $(id_2, es_2)$ has to be computed. Since abstract IDs from $\mathcal{I}^{\#}$ are not reasonably comparable, this leads to a notion of "unknown thread ID" or "any possible thread ID" which we want to avoid.\\
  It is much more reasonable to run the thread ID analysis partially context sensitively, where contexts are differentiated with respect to the current $id$ but not the set of encountered create edges $es$. Computing the least upper bound of sets of encountered create edges is reasonably possible by taking the union of the sets. With these insights we can perform the thread ID analysis partially context sensitively with
  \[\mathbb{C}_\textsf{tID} = \mathcal{I}^{\#} \text{ and } \textsf{context}^{\#}_\textsf{tID}\ (id, es) = id\]
  When we now run this partially context-sensitive analysis, it is possible to encounter cases where precision is lost similarly to how we described it in \autoref{sec:precisionLoss}. We show this with the example program from \autoref{fig:example_thread}. In this program, \texttt{procedure()} is called two times, once without a thread being created beforehand and once with a thread created. For now this thread created in \autoref{code:create} has the unique ID $([main \cdot \langle u_2, s_{tproc} \rangle])$, assuming the program point before the \texttt{create(tproc)} is $u_2$ and $s_{tproc}$ is the start point of the thread procedure \texttt{tproc()}. The two entry states for \texttt{procedure()} are $(([main], \emptyset), \emptyset)$ and $(([main], \emptyset), \{\langle u_2, s_{tproc} \rangle\})$, as once the edge $\langle u_2, s_{tproc} \rangle$ was encountered and once it was not. These states both have the same context, i.e., $([main], \emptyset)$, and thus their entry states are joined. The result is the same as the second entry state $(([main], \emptyset), \{\langle u_2, s_{tproc} \rangle\})$. Since we assume for this example, that no thread was created in \texttt{procedure()}, the return state is the same as the joined entry state. With the way, the $\textsf{combine}^{\#}_\textsf{tID}$ function is defined, this state is used for the point after both calls. In particular, it is used for the state after the call in \autoref{code:proc1}. Thus, the \texttt{create()} action in \autoref{code:create} is actually encountered again with the state $(([main], \emptyset), \{\langle u_2, s_{tproc} \rangle\})$. The result of this is, that the thread created here is now identified with a non-unique abstract ID $([main], \{\langle u_2, s_{tproc} \rangle\})$, because this create edge is in the set of encountered create edges.\\
  With this result, some possible race condition are detected, even though in fact there are none in this program. For once, The thread created in \autoref{code:create} is no longer identified with a unique edge. Thus, it is no longer possible to know that only one thread runs with the procedure \texttt{tproc()} and consequentially, only one thread can perform the assignment instructions in \autoref{code:assign2}. Therefore, a data race is found for this assignment.\\
  Furthermore, it is no longer possible to identify, that no thread has been crated before the variable assignment in \autoref{code:assign1}, because $\langle u_2, s_{tproc} \rangle$ is in the set of encountered create edges of the state at this point. Thus, a possible data race is found also for this assignment instruction.\\
  It is worth mentioning here, that the process we described above is adapted to be better understandable for the reader. Following the actual definitions of static analysis, the corresponding system of constraints for the program points is generated. For this system a solution can be computed with a fix-point solver.\\

  \begin{figure}
    \centering
    \begin{subfigure}{.35\textwidth}
      \centering
      \lstinputlisting[escapechar=|, language=C]{../code/04-example_thread.c}
    \end{subfigure}
    \caption{Example program to illustrate the precision loss of partial contexts in the thread ID analysis}
    \label{fig:example_thread}
  \end{figure}

  To combat this loss of precision, we propose a "thread-create" analysis. This analysis checks for each procedure, whether a thread is possibly created between the entry to it and the return. Note that it does not matter, if a thread is created in the procedure itself or in another procedure which the procedure called. The domain we use for this analysis is the set of boolean values $\mathbb{D}_\textsf{tc} = \{\textsf{true}, \textsf{false}\}$. The analysis tracks whether a procedure \textit{may} create a thread. Thus, we encode uncertainty, i.e., "a thread \textit{may} have been created" with $\textsf{true}$. Similar to the taint analysis from \autoref{sec:formalTaint}, this analysis is context insensitive by itself and thus, $\mathbb{C}_\textsf{tc} = \{\bullet\}$.\\
  In conclusion, the state at some program point $\eta_\textsf{tc}\ [v,\bullet]$ answers the question "\textit{May} a thread have been created since the entry of the current procedure up to the node $v$?".\\
  In the following we give the definitions of the analysis functions for this function:
  \begin{align*}
    \textsf{init}^{\#}_\textsf{tc} &= \textsf{false}\\
    [\![ A ]\!]^{\#}_\textsf{tc}\ c &= \left\{ \begin{array}{ll}
      \textsf{true} & \text{if }A \equiv (create(f);)\\
      c & \text{else}
    \end{array} \right. \\
    \textsf{enter}^{\#}_\textsf{tc}\ c &= \textsf{false}\\
    \textsf{combine}^{\#}_\textsf{tc}\ (c_\textsf{cr}, c_\textsf{ce}) &= c_\textsf{cr} \lor c_\textsf{ce}\\
    \textsf{context}^{\#}_\textsf{tc}\ c &= \bullet\\
  \end{align*}
  
  We note that $\textsf{init}^{\#}_\textsf{tc}$ and $\textsf{enter}^{\#}_\textsf{tc}$ return $\textsf{false}$. The reasoning behind this is that when the program begins and when a function is entered, the analysis should start with a state describing that no thread has been created. After a call, the state should be $\textsf{true}$ if a thread was created by the caller before or if one was created by the callee. Thus, the $\textsf{combine}^{\#}_\textsf{tc}$ function performs a locical "or" operation on both states and returns the result.\\
  \\
  With this analysis, we can improve the $\textsf{combine}^{\#}_\textsf{tID}$ of the thread ID analysis. Before, this function was defined as $\textsf{combine}^{\#}_\textsf{tID}\ (D_\textsf{cr}, D_\textsf{ce}) = D_\textsf{ce}$. We change this, so this analysis only returns the callee state when a thread was possibly created in the call and otherwise returns the caller state from before:
  \[\textsf{combine}^{\#}_\textsf{tID}\ (D_\textsf{cr}, D_\textsf{ce}) = \left\{ \begin{array}{ll}
    D_\textsf{ce} & \text{if } \eta_\textsf{tc}\ [e_f,c]\\
    D_\textsf{cr} & \text{else}
  \end{array} \right.\]
  for an edge $(u, f();, v)$ where $c$ is the corresponding context.\\
  This modification allows the analysis to keep the callee state with the respective set of encountered edges. In the example from \autoref{fig:example_thread} this means, that after the call in \autoref{code:proc1}, the state from before with no encountered edges is kept. This results in no race being detected in \autoref{code:assign1}. Furthermore, because the thread created in \autoref{code:create} is known to be unique, no race is detected in \autoref{code:assign2}.

  \section{Implementation}
    We briefly described in this section, how we implemented the thread-create analysis from the previous chapter in \gob\ and how we used it to improve the \texttt{threadId} analysis that already exists in the analyzer.\\
    \\
    Similar to the \texttt{taintPartialContexts} analysis from \autoref{sec:implTaint}, we implement a new module \texttt{threadCreate} that implements the interface seen in \autoref{fig:analysis_interface}. The implementation is very similar to our formal definition of this analysis from the previous chapter. In the following we discuss some noteworthy points:\\

    \paragraph{Domain:}\mbox{}\\
    The Domain \texttt{D} for this analysis only contains the two boolean values, where \texttt{false} stands for "definitely no thread was created" and \texttt{true} for "maybe a thread was created". For this \gob\ provides the \texttt{MayBool} domain implementing this notion with the corresponing ordering.\\
    Similar to the taint analysis, this analysis does not contribute to the overall context of the analyzer, and thus we chose \texttt{Unit} for the module \texttt{C}.

    \paragraph{Analysis functions:}\mbox{}\\ We handle thread creation in two different functions:\\
    One is the \texttt{threadspawn} function, that exists for exactly this purpose. We implement this function to always returns \texttt{true} for the thread create analysis, as this corresponds to a create edge.\\
    The other function where we handle thread creations is the \texttt{special} function. We implement it so that it returns \texttt{true} not only when the special function that is called is a thread creating function, e.g., \texttt{pthread\_create()}, but also when it is an unknown function. Since we know nothing about unknown functions, we cannot exclude that such a function creates a thread. In any other case besides thread creating and unknown functions, the \texttt{special} function propagates the state from before.\\
    \\
    All other analysis functions either return the state from before (e.g., \texttt{assign} and \texttt{return}) or they are implemented exactly as defined in \autoref{sec:threadTheory} (e.g. \texttt{combine} returns the result of a locical "or" on the caller and callee state).

    \paragraph{The \texttt{query} function:}\mbox{}\\
    To allow the new thread-create analysis to broadcast its information to other analyses we add a \texttt{MayThreadCreate} query. This query is answered by the thread-create analysis with the current state. Other analysis can send a \texttt{MayThreadCreate} query to gain the information, whether a thread has been created from the beginning of the current function up to the point where the query is sent. In the case, that the \texttt{threadCreate} analysis is not performed, this query is answered with \texttt{true} by default, i.e., "A thread \textit{may} have been created".\\
    \\
    \paragraph{Improving the \texttt{threadId} analysis}\mbox{}\\
    In \gob\ there exists an implementation of the thread ID analysis we introduced in \autoref{sec:threadTheory}, namely the \texttt{threadId} analysis. This analysis was always performed context-sensitively. Thus, we add the option \texttt{ana.thread.context.createEdges} to the analyzer. This option is enabled by default, but it allows for removing the set of encountered create edges from the context when it is disabled. We implement this by changing the \texttt{context} function of the \texttt{threadId} analysis, so that it checks this option.\\
    To implement our proposed improvements, we change the \texttt{combine} function of the \texttt{threadId} analysis according to our proposal from \autoref{sec:threadTheory}. It now sends a \texttt{MayThreadCreate} and returns the caller state when the answer is \texttt{false}. Otherwise, the callee state is returned like before our changes.

