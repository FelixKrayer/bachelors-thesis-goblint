% !TeX root = ../main.tex
% Add the above to each chapter to make compiling the PDF easier in some editors.

\chapter{Combatting the Precision Loss of Thread Analyses}\label{chapter:precisionLossThreadAnalyses}
% local traces
  To (((Something))) M. Schwarz et al. \parencite{schwarz2023clustered}.
  \section{Theory}
  In Chapter 6 of their paper \parencite{schwarz2023clustered} they propose an analysis that identifies threads by their creation history. Among other things, this analysis helps to identify which actions may or may not happen in parallel.\\
  As mentioned, threads are identified by their creation history, which is used as an ID to identify different threads. This history is a sequence of create edges starting with $\textsf{main}$. 
  To prevent such a history to grow to infinity, they define the notion of non-unique thread IDs which may identify multiple threads each.\\
  Formally, the set of possible abstract thread IDs $\mathcal{I}^{\#}$ is $(\textsf{main}\cdot\mathcal{P}^*) \times 2^{\mathcal{P}}$, where $\mathcal{P}$ is the set of create edges and $\mathcal{P}^*$ a sequence of such edges. $\langle u, f \rangle \in \mathcal{P}$ refers to an outgoing edge from program point $u$ which creates a thread starting at $f$.
  In this notion, IDs of the form $(i, \emptyset) \in \mathcal{I}^{\#}$ are unique, while $(i, s) \in \mathcal{I}^{\#}$ are not unique if $s \neq \emptyset$.\\
  To better provide a better understanding of these definitions we explain them with the following example:\\
  %TODO: Example
  \\
  \\
  % TODO: Precision Loss
  \\
  We propose a "thread-create" analysis that checks for each function, whether a thread is possibly created between the entry to it and the return. Note that it does not matter, if a thread is created in the function itself or in another function which the function called. The domain we use for this analysis is the set of boolean values $\mathbb{D}_\textsf{tc} = \{\textsf{true}, \textsf{false}\}$. The analysis tracks whether a function \textit{may} create a thread. Thus, we encode uncertainty, i.e., "a thread \textit{may} have been created" with $\textsf{true}$. Therefore, the state at some program point $\eta_\textsf{tc}\ [v,\bullet]$ answers the question "\textit{May} a thread have been created since the entry of the current function up to the node $v$". 
  
  

  \section{Implementation}
    In this section we describe briefly, how we implemented the thread-create analysis we designed in the previous chapter in the \gob\ analyzer and how we used it to improve the \texttt{threadId} analysis that already exists in the analyzer.\\
    Similar to the \texttt{taintPartialContexts} analysis from \autoref{sec:implTaint}, we implement a new module \texttt{threadCreate} that implements the interface seen in \autoref{fig:analysis_interface}.

