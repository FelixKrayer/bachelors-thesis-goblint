% !TeX root = ../main.tex
% Add the above to each chapter to make compiling the PDF easier in some editors.

\chapter{Conclusions}\label{chapter:conclusions}
  In this thesis we introduced static analysis with systems of constraints. After expanding the approach to interprocedural analysis with various kinds of context-sensitivity, we identified a source of precision loss that can occur with context-insensitive and partially context-sensitive analyses. In the following two chapters we focused on two different kinds of analyses, where this loss of precision occurs and proposed an improvement that reduces the precision lost for both kinds. We implemented both of our proposed approaches in the \gob\ analyzer. Afterwards we tested and benchmarked our implementation.\\
  \\
  TODO: Conclusions\\
  \\
  TODO: Future work

\begin{itemize}
%  \item \textbf{Summary:}
%  \item Source of Precison loss identified
%  \item Two ideas for reducing the precision loss for two types of analyses
%  \item implemented in \gob\ and tested to prove viability
%  \item benchmarked to check if a noticeable improvement is achieved
  \item \textbf{Conclusions:}
  \item -> in general insens not that costly (precision) but also not that much faster.
  \item -> however insens produces less errors (Timeout/Stack overflow)
  \item -> taint in general not that much benefit, but it has its uses (but not much faster compared to sens)
  \item recursive is problematic with sens (error often), insens finds unknown much faster.
  \item \textbf{Future Work}
  \item general approach (identify partial information of a state that has not been changed by call and keep that from the caller state instead of overwriting it with less precis callee info)
  \item more extensive benchmark for thread Create
  \item inspect types of programs (e.g. recursive)
  \item can be combined with any of related work approaches without any issues
  \item autotuner somehow???
\end{itemize}

